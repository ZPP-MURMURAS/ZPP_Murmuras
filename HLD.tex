\documentclass[12pt]{article}

\usepackage{hyperref}
\usepackage[T1]{fontenc}
\usepackage[polish]{babel}
\usepackage{amssymb}
\usepackage{amsmath}
\usepackage[utf8]{inputenc}
\usepackage{lmodern}
\selectlanguage{polish}

\title{ZPP Murmuras - HLD}
\author{Gustaw Blachowski \and Szymon Kozłowski \and Natalia Junkiert \and Kamil Dybek}
\date{}

\begin{document}

\maketitle

\section*{Wprowadzenie}

Celem projektu jest stworzenie uniwersalnego rozwiązania do procesowania danych uzyskanych z ekranu smartfonu (wspierany system: Android 9 wzwyż), takich jak lokalizacja elementu na ekranie, jego typ (tekst, zdjęcie itp), jego zawartość. Dane te reprezentują zawartość widzianą przez użytkownika - posty na mediach społecznościowych, reklamy na witrynach, itp. System ten ma umożliwić analizę informacji dotyczących treści obserwowanych przez użytkowników, wspierając tym samym badania o charakterze komercyjnym i społecznym. \\

Przykładowymi zastosowaniami danych przetowrzonych przez system są (1) analiza danych o oglądanych przez użytkownika reklamach oraz (2) badanie poglądów politycznych i społecznych. \\

(1) System będzie umożliwiał wygodne badanie zasięgów kuponów promocyjnych. Dzięki temu będziemy mogli sprawdzić, jak dany kupon radzi sobie na tle innych, oraz jak radzi sobie konkurencja. Może to być pomocne w planowaniu przyszłych akcji marketingowych. \\

(2) System będzie umożliwiał lokalną analizę danych prywatnych takich jak konwersacje użytkowników. Dostarczy to wiarygodnych danych trudno dostępnych innymi metodami, na przykład poglądów politycznych i społecznych.


\section*{Istniejące rozwiązania}
DOZRO

\section*{Rozwiązanie Murmurasa}
Istniejące prototypowe rozwiązanie korzysta z danych dostarczonych w postaci tekstu wykrytego na zrzutach ekranu w połączeniu z metadanymi, takimi jak lokalizacja pola tekstowego na ekranie. Następnie dane te są poddawane bardzo podstawowej obróbce (usuwanie pustych kolumn itp), a następnie są przetwarzanie przez ChatGPT4o-mini (za pomocą prompt-engineeringu). Rozwiązanie to ma dwa zasadnicze problemy: nie działa ono lokalnie na urządzeniu mobilnym (model jest za duży), a dane są często opisywane niepoprawnie (np. objętość jest traktowana jak cena produktu).

\section*{Specyfikacja skończonego projektu}
1. Wymagane jest narzędzie, które przeprocesuje dane wyekstraktowane z urządzenia w postać nadającą się do użycia przez model. \\
2. Wymagane jest użycie narzędzia z dziedziny Machine Learningu do ekstrakcji interesujących z naszej perspektywy danych. \\
3. Opcjonalne jest narzędzie do postprocessingu danych wyjściowych narzędzia z punktu 2 do wspólnego formatu. \\
4. Wymagany jest deployment powyższych trzech narzędzi na urządzenie mobilne.

\section*{Wyzwania}

\subsection*{Zapewnianie prywatności użytkowników}
DOZRO (do doprezyzowania z Murmurasem;)

\subsection*{Wymagania sprzętowe}
Zarówno współczesne LLMy jak i algorytmy preprocessingu danych wymagają często dużej ilości zasobów; jednoczesnie chcemy aby wszystko działało lokalnie na urzadzeniu mobilnym. Wyzwaniem więc będzie dobór narzędzi które nie będą zbyt zasobożerne.

\subsection*{Benchmarkowanie}
W celu oceny jakości naszego rozwiązania obecnie planujemy posługiwać się benchmarkiem zapewnionym nam przez Murmuras. Bazuje on na obliczaniu funkcji podobieństwa między wynikiem użytego modelu a wynikiem modelu wzorcowego (obecnie jest to GPT4o-mini). Benchmark ten może okazać się niewystarczająco dokładny i miarodajny także może pojawić się konieczność zaproponowania alternatywy, przykładowo testowania systemu na sztucznie wygenerowanych i poetykietowanych danych.

\section*{Propozycja rozwiązania}
W implementacji naszego rozwiązania wyróżniamy następujące 4 główne moduły: \\
1. Preprocessing danych:
Na ten moment rozpatrujemy trzy możliwe rozwiązania: użycie rozwiązań z zakresu uczenia maszynowego, w szczególności klastryzacji, algorytmów niezwiązanych z MLem lub pominięcie jakiegokolwiek preprocessingu. \\
2. Processing danych:
W tym celu wykorzystamy prawdopodobnie wybrany LLM, ale technicznie rzecz biorąc nie jesteśmy w tym temacie ograniczeni. Na tem moment prawdopodobnie będziemy korzostać z narzędzi HuggingFace; zapewniają one proste i wygodne użycie wielu ogólnodostępnych modeli. \\ 
3. Postprocessing:
DOZRO \\
4. Deployment na urządzeniu mobilnym:
Stworzymy aplikację mobilną bądź dodamy funkcjonalność do istniejącej w ramach której zaimplementujemy powyższe punkty. Utworzymy service działający w tle i przetwarzający nadpływające dane w czasie rzeczywistym. Planujemy wykorzystać framework TensorFlow Lite (TensorFlow dla urządzeń mobilnych) ze względu na łatwą integrację z aplikacjami rozwijanymi w Android Studio. \\
DOZRO - gdzie wysyłamy zebrane dane

\end{document}
