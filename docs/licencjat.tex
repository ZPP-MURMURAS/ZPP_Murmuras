%
% Niniejszy plik stanowi przykład formatowania pracy magisterskiej na
% Wydziale MIM UW.  Szkielet użytych poleceń można wykorzystywać do
% woli, np. formatujac wlasna prace.
%
% Zawartosc merytoryczna stanowi oryginalnosiagniecie
% naukowosciowe Marcina Wolinskiego.  Wszelkie prawa zastrzeżone.
%
% Copyright (c) 2001 by Marcin Woliński <M.Wolinski@gust.org.pl>
% Poprawki spowodowane zmianami przepisów - Marcin Szczuka, 1.10.2004
% Poprawki spowodowane zmianami przepisow i ujednolicenie 
% - Seweryn Karłowicz, 05.05.2006
% Dodanie wielu autorów i tłumaczenia na angielski - Kuba Pochrybniak, 29.11.2016

% dodaj opcję [licencjacka] dla pracy licencjackiej
% dodaj opcję [en] dla wersji angielskiej (mogą być obie: [licencjacka,en])
\documentclass[licencjacka,en]{pracamgr}
\usepackage{hyperref}  % Enables clickable links
\usepackage{xcolor}    % Allows hyperlink color customization

% Set hyperlink colors
\hypersetup{
    colorlinks=false,
    urlcolor=blue
}

% Dane magistranta:
\autori{Szymon Kozłowski}{448304}
\autorii{Gustaw Blachowski}{448194}
\autoriii{Kamil Dybek}{448224}
\autoriv{Natalia Junkiert}{448267}

\title{Innovative methods of processing data coming from mobile devices for market and scientific research}

\tytulang{Innovative methods of processing data coming from mobile devices for market and scientific research}
\titlepl{Innowacyjne metody przetwarzania danych pochodzących z urządzeń mobilnych na potrzeby badań rynkowych i naukowych}

%kierunek: 
% - matematyka, informacyka, ...
% - Mathematics, Computer Science, ...
\kierunek{Computer Science}

% informatyka - nie okreslamy zakresu (opcja zakomentowana)
% matematyka - zakres moze pozostac nieokreslony,
% a jesli ma byc okreslony dla pracy mgr,
% to przyjmuje jedna z wartosci:
% {metod matematycznych w finansach}
% {metod matematycznych w ubezpieczeniach}
% {matematyki stosowanej}
% {nauczania matematyki}
% Dla pracy licencjackiej mamy natomiast
% mozliwosc wpisania takiej wartosci zakresu:
% {Jednoczesnych Studiow Ekonomiczno--Matematycznych}

% \zakres{Tu wpisac, jesli trzeba, jedna z opcji podanych wyzej}

% Praca wykonana pod kierunkiem:
% (podać tytuł/stopień imię i nazwisko opiekuna
% Instytut
% ew. Wydział ew. Uczelnia (jeżeli nie MIM UW))
\opiekun{Jacek Sroka PhD\\
  Institute of Informatics\\
  }

% miesiąc i~rok:
\date{\today}

%Podać dziedzinę wg klasyfikacji Socrates-Erasmus:
\dziedzina{ 
%11.0 Matematyka, Informatyka:\\ 
%11.1 Matematyka\\ 
%11.2 Statystyka\\ 
%11.3 Informatyka\\ 
11.4 Artificial Intelligence\\ 
%11.5 Nauki aktuarialne\\
%11.9 Inne nauki matematyczne i informatyczne
}

%Klasyfikacja tematyczna wedlug AMS (matematyka) lub ACM (informatyka)
\klasyfikacja{
  I.2.7: Natural Language Processing\\
  H.3.3: Information Search and Retrieval}

% Słowa kluczowe:
\keywords{LLM, NLP, BERT, Android, Edge-device, Fine-Tuning}

% Tu jest dobre miejsce na Twoje własne makra i~środowiska:
\newtheorem{defi}{Definicja}[section]

% koniec definicji

\begin{document}

\maketitle

%tu idzie streszczenie na strone poczatkowa
\begin{abstract}
In an era of rapidly evolving digital applications, traditional scraping techniques face increasing challenges in maintaining reliable data collection pipelines. Commissioned by Murmuras, a company specializing in commercial and scientific data analysis \cite{murmuras}, this project presents a novel approach to processing phone screen content, such as social media posts and website advertisements. Our solution leverages Large Language Models (LLMs) running locally on the user's device to handle diverse data formats while ensuring that sensitive information remains protected. The primary application explored in this study is the extraction of discount coupons, demonstrating the feasibility of our method in identifying and structuring valuable content from varying digital sources. Furthermore, the system is designed to be easily adaptable to other use cases, such as analyzing users' political views. The results highlight the potential of LLM-driven content analysis as an alternative to conventional scraping techniques.
\raggedright
\end{abstract}

\tableofcontents
%\listoffigures
%\listoftables

\chapter{Introduction}
\section{Project background and motivation}
With the rapid advancement of information technology, the internet has become one of the most crucial facets for many businesses to perform marketing activities \cite{design_of_coupons}. One of the key marketing tools in business-to-consumer (B2C) e-commerce is the digital coupon (also referred to as an electronic coupon) \cite{targeted_reminders}. In comparison to paper coupons, digital coupons are characterized by their wide reach, rapid distribution, and low spread costs. Furthermore, a key advantage of digital coupons is their ability to facilitate targeted marketing by offering personalized discounts to different customers, thereby increasing sales \cite{design_of_coupons}. To maximize the benefits of digital coupons, it is essential for businesses to assess the effectiveness of their coupon campaigns, evaluate their reach, and analyze their competitors’ strategies. By tracking key performance metrics such as redemption rates, customer engagement, and sales impact, businesses can refine their marketing approaches to optimize results. Additionally, studying competitors' digital coupon strategies enables businesses to identify market trends, adjust their promotional tactics, and maintain a competitive edge in the evolving digital marketplace. 

Machine learning has rapidly become a central focus in computer science research, offering powerful capabilities in pattern recognition and information extraction from unstructured data. This advancement has led to the development of models that can learn relevant features from large datasets, reducing reliance on heuristic-based algorithms that require extensive parameter tuning and handcrafted rules. Such models are particularly effective in handling the variability inherent in real-world data \cite{ml_general}, including diverse coupon designs.

Recent statistics underscore the significance of mobile devices in this domain. For example, studies have shown that over 90\% of digital coupon users access their vouchers via smartphones \cite{emarketer_coupon_stats}, and similar figures are reported by other industry sources \cite{coupon_stats_2}. This high rate of mobile usage creates a pressing need for coupon analysis tools that are optimized for mobile platforms, ensuring that consumers receive timely and personalized offers regardless of their location or device.

In light of these trends, the company Murmuras has tasked us with developing a machine learning model that can be deployed as a mobile application. This model will process input representing the user's onscreen view and extract digital coupons along with their relevant data. This solution must be capable of running locally on the device, ensuring efficient processing without relying on external servers. By leveraging advanced machine learning techniques, the app will handle the diverse formats and layouts of digital coupons, thereby facilitating the collection of data regarding coupons.

\section{The definition of a coupon} 
A coupon is a physical piece of paper or digital voucher that can be redeemed for a financial discount when purchasing a product \cite{coupon_definition}. A coupon is characterized by a name, expiration date, and a discount type (e.g. '20\% off', 'buy 1 get 1 free', etc.), however, not every coupon contains each of these features. Furthermore, coupons may contain numerous other features such as images and eligibility requirements. Henceforth, the term 'coupon' will refer exclusively to a digital coupon.

\section{Project goals}
\begin{enumerate}
    \item A tool to process the data extracted from the device into a format suitable for use by the model.
    \item A machine learning tool for extracting the data that is of interest to us, such as the coupon name, expiration dates, prices, etc. The model should be capable of handling various coupon formats and layouts with high accuracy.
    \item An optional tool for post-processing the output data from the tool mentioned in the previous point into a common format.
    \item An application that runs the above three tools on a mobile device. (Optional)
    \item A key requirement is that the machine learning model must be deployable on the mobile device itself to guarantee data privacy.
\end{enumerate} 

\section{Potential applications of the project}
\subsection{Assessing coupon effectiveness}
Our solution will aid businesses in analyzing consumer behaviour and optimizing their marketing strategies accordingly. By facilitating the collection of data on coupon characteristics and their redemption rates, businesses will be able to assess the effectiveness of their coupon campaigns—determining whether they reach the intended audience and achieve the desired results. Additionally, large-scale analysis of coupon data can reveal valuable insights into purchasing patterns, preferred discount types, and the most appealing products or services for different customer segments. With this information, businesses can refine their promotional strategies, tailor offers to specific demographics, and enhance overall customer engagement.

\subsection{Market analysis and competitor monitoring}
The aforementioned gathering of data can also be utilized to monitor competitors' coupon strategies, their effectiveness, and whether they provide better discounts. Using machine learning to identify and analyze competitors' strategies is more cost-effective compared to exhaustive web scraping or mystery shopping \cite{competitor_tariffs}. This will enable businesses to make better-informed decisions about their own marketing campaigns and provide a comprehensive understanding of the competitive landscape.

\chapter{Machine learning and the dangers associated with it}
Note: this chapter is a work in progress, bullet points aim to provide guidance when writing this section

% \subsection{Benchmark}
% Benchmarking is the process of running a set of, among others, computer programs against a set of tests to assess their relative performance or precision % \cite{benchmark}.
\begin{enumerate}
    \raggedright
    \item \textbf{Difference between Machine Learning, Artificial Intelligence, and Deep Learning}
    \begin{enumerate}
        \item \textbf{Artificial Intelligence (AI):} AI is a broad field of computer science focused on creating systems that can perform tasks typically requiring human intelligence, such as reasoning, problem-solving, and decision-making.
        \item \textbf{Machine Learning (ML):} ML is a subset of AI that involves training algorithms on data to enable systems to learn patterns and make predictions or decisions without being explicitly programmed.
        \item \textbf{Deep Learning (DL):} DL is a specialized subset of ML that uses artificial neural networks, particularly deep neural networks, to model complex patterns and perform tasks like image and speech recognition.
    \end{enumerate}

    \item \textbf{Benchmark}
    \begin{enumerate}
        \item \textbf{Definition:} Benchmarking is the process of running a set of programs against predefined tests to assess their relative performance or precision.
        \item \textbf{Usage:} Benchmarks are used to evaluate and measure the efficiency, accuracy, and effectiveness of machine learning models, AI systems, or computing hardware against established standards or competing solutions.
    \end{enumerate}

    \item \textbf{Understanding ML Models}
    \begin{enumerate}
        \item \textbf{What is a Model?} A model in machine learning is a mathematical representation of a system that learns from data to make predictions or decisions.
        \item \textbf{How a Model Works:}
        \begin{enumerate}
            \item \textbf{Training:} A model is trained using datasets that contain input-output pairs.
            \item \textbf{Datasets:} Collections of labeled or unlabeled data used for training and evaluation.
            \item \textbf{Linear Regression:} A basic ML model that finds relationships between input and output variables.
            \item \textbf{Supervised vs. Unsupervised Learning:} Supervised learning uses labeled data, while unsupervised learning finds hidden patterns in unlabeled data.
            \item \textbf{Federated Learning:} A decentralized ML approach where models are trained across multiple devices without sharing raw data.
            \item \textbf{Computer Vision:} A field of AI focused on enabling machines to interpret visual data.
        \end{enumerate}
        \item \textbf{Quantization:} A technique for reducing the computational cost and memory footprint of ML models by representing numerical values with lower precision.
    \end{enumerate}

    \item \textbf{What is NLP?}
    \begin{enumerate}
        \item \textbf{Definition and Importance:} Natural Language Processing (NLP) is a branch of AI focused on enabling computers to understand, interpret, and generate human language.
        \item \textbf{NLP Models:}
        \begin{enumerate}
            \item \textbf{BERT:} A transformer-based model designed for contextual word understanding.
            \item \textbf{LLaMA:} A language model optimized for efficiency in various NLP tasks.
            \item \textbf{ChatGPT:} A generative AI model designed for conversational applications.
            \item \textbf{Comparison:} These models differ in size, memory usage, training methods, and suitability for different tasks.
        \end{enumerate}
    \end{enumerate}

    \item \textbf{Should We Be Scared of AI?}
    \begin{enumerate}
        \item \textbf{Reference:} \textit{YouTube Video} \url{https://www.youtube.com/watch?v=yh1pF1zaauc} (from our mentor).
        \item \textbf{Privacy and Ethics:} 
        \begin{enumerate}
            \item \textbf{Problem:} Concerns about data collection and processing.
            \item \textbf{Solution:} Our project processes data locally to mitigate privacy risks.
        \end{enumerate}
        \item \textbf{Adversarial Attacks:} AI vulnerabilities where manipulated inputs lead to incorrect outputs.
        \item \textbf{Accuracy Concerns:} 
        \begin{enumerate}
            \item \textbf{Reliability:} How can we ensure our model is correct?
            \item \textbf{Lack of Human Oversight:} Risks of AI decision-making without supervision.
        \end{enumerate}
        \item \textbf{Environmental Concerns:} 
        \begin{enumerate}
            \item \textbf{Resource Usage:} AI training requires high computational power.
            \item \textbf{Optimization:} Fine-tuning instead of full training can reduce energy consumption.
        \end{enumerate}
    \end{enumerate}

\end{enumerate}

\chapter{Overview of the existing solutions}
To our knowledge, as of writing this thesis, there are no publicly available solutions that directly address this problem. The most comparable approaches involve existing multimodal models. While widely used models like ChatGPT and Gemini offer some relevant capabilities, they are not highly precise for this specific task. A major limitation of such models is their large size—for instance, GPT-3 has 175 billion parameters\cite{chatgpt_params}—making them impractical for mobile deployment \cite{LinguaLinked}.

Alternatively, Computer Vision models exist for extracting text and bounding boxes from screen images. Microsoft’s OmniParser \cite{omniparser_intro}, for example, performs well in this area but still requires preprocessing similar to our approach. Moreover, our experiments running OmniParser locally indicate that it depends on CUDA technology, making it unsuitable for mobile deployment.

\section{Murmuras' existing solution} 
Murmuras' current approach relies on fixed scripts tailored to specific applications, making it inflexible and difficult to generalize across diverse coupon formats. This lack of adaptability limits its usefulness in real-world scenarios where coupon structures vary widely. Since our goal is to develop a solution that is easily adaptable for processing diverse mobile content, this method is not well-suited for our needs.

In contrast, Murmuras' most-recent proof of concept involves basic preprocessing of the extracted data before sending it to GPT-4o-mini for further processing. This approach leverages an LLM to interpret the data to extract relevant coupon details. However, the reliance on an external server means the solution does not run locally on the mobile device, leading to potential privacy concerns, latency issues, and a dependence on internet connectivity. 

Additionally, the accuracy of this method is suboptimal. According to their own benchmarks, the average similarity ratio is only 56.49\%, indicating significant inconsistencies in the extracted data. This benchmark measures the accuracy of extracted coupons by comparing them to expected ground-truth values across five key attributes: product name, discount percentage, old price, new price, and coupon validity.

The evaluation process involves using difflib’s SequenceMatcher, a text similarity algorithm that computes the ratio of matching sequences between two strings \cite{sequence_matcher}. For each extracted coupon, the similarity scores of its five attributes are individually compared against the ground-truth values. The results are then averaged to generate an overall similarity percentage. A low similarity ratio indicates significant variations between the extracted and expected coupon details, highlighting challenges in precise text interpretation and extraction. 

\section{Scapegraph AI}
ScrapeGraphAI is an open-source library that streamlines data extraction from websites and local documents by utilizing LLMs and graph logic to construct scraping pipelines \cite{scapegraph_repo}. The library supports integration with various LLMs, including local models through the use of Ollama \cite{ollama_repo} \cite{scapegraph_usage}.

However, Scrapepraph AI provides only Python and Node.js SDKs \cite{scapegraph_sdks}, which could prove to be an issue with regard to mobile deployment, because neither Python nor Node.js is natively supported on iOS or Android \cite{android_dev_site} \cite{ios_dev_site}.

Moreover, due to mobile devices typically having limited processing power and memory compared to desktop computers or servers \cite{mobile_resources}, we cannot solely rely on the size of the model in order to improve performance. We believe that through fine-tuning LLMs, we are able to develop tools that are far more viable for edge device usage.

% \section{}

% \chapter{Technologies}
% \chapter{Architecture design}
% \chapter{Performance}
% \chapter{Possible extensions}
% \chapter{Summary}
% \chapter{Charts}

\begin{thebibliography}{99}

\addcontentsline{toc}{chapter}{Bibliography}
\raggedright

\bibitem{murmuras} 
\textit{Murmuras website}.  
\url{https://murmuras.com/}.  
[Accessed 2025-02-11].

\bibitem{coupon_definition}
\textit{Britannica Dictionary definition of COUPON}.
\url{https://www.britannica.com/dictionary/coupon}.
[Accessed 2025-02-03].

% \bibitem{benchmark} 
% \textit{Computer Benchmark}.
% \url{https://bhatabhishek-ylp.medium.com/benchmarking-in-computer-c6d364681512}.
% [Accessed 2025-02-03].

\bibitem{design_of_coupons}
Xiong Keyi, Yang Wensheng
\textit{Research on the Design of E-coupons for Directional Marketing of Two Businesses in Competitive Environment}.
\url{https://www.sciencepublishinggroup.com/article/10.11648/j.ijefm.20200801.16}.
[Accessed 2025-02-04].

\bibitem{targeted_reminders}
Li Li, et. al.
\textit{Targeted reminders of electronic coupons: using predictive analytics to facilitate coupon marketing}.
\url{https://link.springer.com/article/10.1007/s10660-020-09405-4}.
[Accessed 2025-02-04].

\bibitem{competitor_tariffs}
Bernhard König, et. al.
\textit{Analysing competitor tariffs with machine learning}.
\url{https://www.milliman.com/en/insight/analysing-competitor-tariffs-with-machine-learning}.
[Accessed 2025-02-04].

\bibitem{ml_general}
Iqbal H. Sarker
\textit{Machine Learning: Algorithms, Real-World Applications and Research Directions}.
\url{https://link.springer.com/article/10.1007/s42979-021-00592-x}.
[Accessed 2025-02-05].

\bibitem{emarketer_coupon_stats}
Sara Lebow
\textit{How consumers access digital coupons}.
\url{https://www.emarketer.com/content/how-consumers-access-digital-coupons}.
[Accessed 2025-02-05].

\bibitem{coupon_stats_2}
\textit{Unveiling IT Coupons Trends and Statistics}
\url{https://www.go-globe.com/unveiling-it-coupons-trends-statistics/}.
[Accessed 2025-02-05].

\bibitem{chatgpt_params}
\textit{Tom B. Brown, Benjamin Mann, Nick Ryder, Melanie Subbiah, Jared
Kaplan, Prafulla Dhariwal, Arvind Neelakantan, Pranav Shyam, Girish
Sastry, Amanda Askell, Sandhini Agarwal, Ariel Herbert-Voss, Gretchen
Krueger, Tom Henighan, Rewon Child, Aditya Ramesh, Daniel M.
Ziegler, Jeffrey Wu, Clemens Winter, Christopher Hesse, Mark Chen,
Eric Sigler, Mateusz Litwin, Scott Gray, Benjamin Chess, Jack Clark,
Christopher Berner, Sam McCandlish, Alec Radford, Ilya Sutskever, and
Dario Amodei. Language models are few-shot learners, 2020}

\bibitem{scapegraph_repo}
Marco Perini, Lorenzo Padoan, Marco Vinciguerra
\textit{Scrapegraph-ai}.
\url{https://github.com/VinciGit00/Scrapegraph-ai}.
[Accessed 2025-02-24].

\bibitem{ollama_repo}
\textit{Ollama}.
\url{https://github.com/ollama/ollama}.
[Accessed 2025-02-24].

\bibitem{scapegraph_usage}
Marco Perini, Lorenzo Padoan, Marco Vinciguerra
\textit{Scrapegraph-ai usage}.
\url{https://github.com/ScrapeGraphAI/Scrapegraph-ai?tab=readme-ov-file#-usage}.
[Accessed 2025-02-24].

\bibitem{scapegraph_sdks}
Marco Perini, Lorenzo Padoan, Marco Vinciguerra
\textit{Scrapegraph-ai API and SDKs}.
\url{https://github.com/ScrapeGraphAI/Scrapegraph-ai?tab=readme-ov-file#-scrapegraph-api--sdks}.
[Accessed 2025-02-24].

\bibitem{android_dev_site}
\textit{Android developer fundamentals website}.
\url{https://developer.android.com/guide/components/fundamentals}.
[Accessed 2025-02-24].

\bibitem{ios_dev_site}
\textit{Apple developer website}.
\url{https://developer.apple.com/develop/}.
[Accessed 2025-02-24].

\bibitem{omniparser_intro}
\textit{Yadong Lu, Jianwei Yang, Yelong Shen, and Ahmed Awadallah. Omni-
parser for pure vision based gui agent, 2024.}

\bibitem{mobile_resources}
Xiang Li, et. al.
\textit{Large Language Models on Mobile Devices: Measurements, Analysis, and Insights}
\url{https://dl.acm.org/doi/10.1145/3662006.366205}

\bibitem{LinguaLinked}
Junchen Zhao, et. al.
\textit{LinguaLinked: A Distributed Large Language Model Inference System for Mobile Devices}
\url{https://arxiv.org/pdf/2312.00388}

\bibitem{sequence_matcher}
\textit{difflib — Helpers for computing deltas}
\url{https://docs.python.org/3/library/difflib.html}

\end{thebibliography}

\end{document}


%%% Local Variables:
%%% mode: latex
%%% TeX-master: t
%%% coding: latin-2
%%% End:
